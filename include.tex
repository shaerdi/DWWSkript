
%% Always need these
\usepackage[utf8]{inputenc}
\usepackage[pagebackref,hyperindex=true]{hyperref} % load hyperref before algorithm!!!!!!!
\usepackage{algorithm}
\usepackage{algorithmic}
\usepackage{amsmath}
\usepackage{fancyhdr}
\usepackage{graphicx}
\usepackage{siunitx}

%% Set figure path
\graphicspath{figures/}
\DeclareGraphicsExtensions{.pdf,.png,.jpg}

%% Colored links in pdf
\usepackage{xcolor}
%\newcommand{\blue}{ \color{blue} }

%% Bibliography text
% nicer backref links
\renewcommand*{\backref}[1]{}
\renewcommand*{\backrefalt}[4]{%
\ifcase #1 %
(Not cited.)%
\or
(Cited on page~#2.)%
\else
(Cited on pages~#2.)%
\fi}
\renewcommand*{\backrefsep}{, }
\renewcommand*{\backreftwosep}{ and~}
\renewcommand*{\backreflastsep}{ and~}

%% Language
\usepackage[ngerman,english]{babel}
\selectlanguage{ngerman}

%% Fonts
\usepackage[T1]{fontenc}
\usepackage{amsfonts} % need for mathbb
\usepackage[font=footnotesize,labelfont=bf]{caption} % small captions
%
%% Subfigures
\usepackage[lofdepth,lotdepth]{subfig}

%% Code
\usepackage{listings}
\lstset{ %
  language=matlab, % choose the language of the code
  basicstyle=\small\ttfamily, % the size of the fonts that are used for the code
  numbers=left, % where to put the line-numbers
  numberstyle=\small\ttfamily\color[rgb]{0.6,0.6,0.6}, % the size of the fonts that are used for the line-numbers
  stepnumber=1, % the step between two line-numbers. If it's 1 each line
  xleftmargin=4mm,
  % will be numbered
  numbersep=5pt, % how far the line-numbers are from the code
  backgroundcolor=\color{white}, % choose the background color. You must add \usepackage{color}
  showspaces=false, % show spaces adding particular underscores
  showstringspaces=false, % underline spaces within strings
  showtabs=false, % show tabs within strings adding particular underscores
%  frame=l, % adds a frame around the code
  frame=single,
  tabsize=2, % sets default tabsize to 2 spaces
  breaklines=true, % sets automatic line breaking
  breakatwhitespace=false, % sets if automatic breaks should only happen at whitespace
  % also try caption instead of title
  escapeinside={@}{@}, % if you want to add a comment within your code
  morekeywords={*,...}, % if you want to add more keywords to the set
  keywordstyle=\color[rgb]{0,0,1},
  commentstyle=\color[rgb]{0.133,0.545,0.133}\textit,
  stringstyle=\color[rgb]{0.627,0.126,0.941},
}


% Nomenclature
\usepackage[refpage, german]{nomencl}
\makenomenclature
\renewcommand{\pagedeclaration}[1]{\dotfill\makebox[4em][l]{Seite #1}}

%% Customizations
\newenvironment{packed_item}{
\vspace{-2mm}
\begin{itemize}
  \setlength{\itemsep}{1.5pt}
  \setlength{\parskip}{0pt}
  \setlength{\parsep}{0pt}
}{\vspace{-2mm}\end{itemize}}

\newenvironment{abstr}{%
\vfill\small%
\begin{center}%
{\bfseries \abstractname}%
\end{center}%
\quotation}%
{\vfill}%

% Provide the \*matter commands in non-book classes
\providecommand{\frontmatter}{%
\clearpage%
\pagenumbering{roman}%
\setcounter{page}{1}}%

\providecommand{\mainmatter}{%
\setlength{\headheight}{45px}%
\clearpage%
\pagenumbering{arabic}%
\setcounter{page}{1}%
\pagestyle{fancy}%
\fancyhf{}%
\fancyhead[L]{Kurs DWW}%
\fancyhead[R]{\includegraphics[height=40px]{figures/NTB-FHO_LOGO_2.png}}%
\fancyfoot[R]{\thepage}%
}%

\providecommand{\backmatter}{%
\clearpage%
\pagenumbering{arabic}%
\setcounter{page}{1}}%

% Paragraph Intendation
\setlength{\parindent}{0pt}
\setlength{\parskip}{2ex}

%% todonotes. gives command \todo{do this thing}
\usepackage{todonotes}

%% Outlines
\usepackage{outlines}

%% TikZ
\usepackage{tikz}
\usetikzlibrary{
    calc,
    shapes,
    arrows,
    decorations.pathmorphing,
    decorations.pathreplacing
}
\usepackage[europeanresistor]{circuitikz}%[americanresistor]
\usepackage{pgfplots}

%% Include macros
%% general vector style (bold or with arrow)
\newcommand{\ve}[1]{
    %\mathbf{#1}
    \vec{#1}
}
\newcommand{\mat}[1]{
    \mathbf{#1}
    %\vec{#1}
}

%% common shortcuts
\providecommand{\argmin}{\operatorname*{argmin}} % operatorname makes _{..} appear centered
\providecommand{\argmax}{\operatorname*{argmax}} % operatorname makes _{..} appear centered
\newcommand{\dd}[1]{\,\mathrm{d}#1} % integration: \int f(x) \dd{x}
\newcommand{\EE}{\mathbb{E}}        % expectation value
\newcommand{\RR}{\mathbb{R}}        % set of real numbers
\newcommand{\CC}{\mathbb{C}}        % set of complex numbers
\newcommand{\NN}{\mathbb{N}}        % set of natural numbers
\newcommand{\OO}{\mathcal{O}}       % big O notation (asymptotic complexity)
\newcommand{\TT}{\mathbb{T}}        % time interval

%% Vectors (lowercase letters)
\renewcommand{\a}{\ve{a}}
\renewcommand{\b}{\ve{b}}
\renewcommand{\c}{\ve{c}}
\renewcommand{\d}{\ve{d}}
\newcommand{\e}{\ve{e}}
\newcommand{\f}{\ve{f}}
\newcommand{\g}{\ve{g}}
\newcommand{\h}{\ve{h}}
\renewcommand{\i}{\ve{i}}
\renewcommand{\j}{\ve{j}}
\renewcommand{\k}{\ve{k}}
\renewcommand{\l}{\ve{l}}
\newcommand{\m}{\ve{m}}
\newcommand{\n}{\ve{n}}
\renewcommand{\o}{\ve{o}}
\newcommand{\p}{\ve{p}}
\newcommand{\q}{\ve{q}}
\renewcommand{\r}{\ve{r}}
\newcommand{\s}{\ve{s}}
\renewcommand{\t}{\ve{t}}
\renewcommand{\u}{\ve{u}}
\renewcommand{\v}{\ve{v}}
\newcommand{\w}{\ve{w}}
\newcommand{\x}{\ve{x}}
\newcommand{\y}{\ve{y}}
\newcommand{\z}{\ve{z}}

%% Matrices (uppercase letters)
\newcommand{\A}{\mat{A}}
\newcommand{\B}{\mat{B}}
\newcommand{\C}{\mat{C}}
\newcommand{\D}{\mat{D}}
\newcommand{\E}{\mat{E}}
\newcommand{\F}{\mat{F}}
\newcommand{\G}{\mat{G}}
\renewcommand{\H}{\mat{H}}
\newcommand{\I}{\mat{I}}
\newcommand{\J}{\mat{J}}
\newcommand{\K}{\mat{K}}
\renewcommand{\L}{\mat{L}}
\newcommand{\M}{\mat{M}}
\newcommand{\N}{\mat{N}}
\renewcommand{\O}{\mat{O}}
\renewcommand{\P}{\mat{P}}
\newcommand{\Q}{\mat{Q}}
\newcommand{\R}{\mat{R}}
\renewcommand{\S}{\mat{S}}
\newcommand{\T}{\mat{T}}
\newcommand{\U}{\mat{U}}
\newcommand{\V}{\mat{V}}
\newcommand{\W}{\mat{W}}
\newcommand{\X}{\mat{X}}
\newcommand{\Y}{\mat{Y}}
\newcommand{\Z}{\mat{Z}}

% Macro for 'List of Symbols', 'List of Notations' etc...
\def\listofsymbols{\input{symbols} \clearpage}
\def\addsymbol #1: #2#3{$#1$ \> \parbox{115mm}{#2 \dotfill \pageref{#3}}\\}
\def\newnot#1{\label{#1}}

\newcommand*{\nom}[3][\empty]{%%% \empty: Standardwert des optionalen Parameters
  \ifthenelse{\equal{#1}{\empty}}%
    {#2\nomenclature{#2}{#3}}%
    {#2\nomenclature[#1]{#2}{#3}}%
}

\newcommand{\changefont}[3]{\fontfamily{#1}\fontseries{#2}\fontshape{#3}\selectfont}
\newcommand{\codeemph}[1]{{\changefont{pcr}{m}{n}#1}}

\newcommand{\definition}[2]{\paragraph{#1:}{#2}}
\newcommand{\bemerkung}[1]{\paragraph{Bemerkung:}#1}

\newcommand{\bsp}[1]{\textbf{#1}}

\newenvironment{eqnarr}{%
\begin{IEEEeqnarray*}{rCl}%
}%
{\end{IEEEeqnarray*}}%

\newcommand*\circled[1]{\tikz[baseline=(char.base)]{
            \node[shape=circle,draw,inner sep=2pt] (char) {#1};}}


\newcommand{\tikzmark}[1]{\tikz[overlay,remember picture] \node (#1) {};}


\definecolor{citecol}{rgb}{0.5,0,0}
\hypersetup
{
bookmarksopen=true,
pdftitle=DWW Skript,
pdfauthor=NTB ICE,
pdfsubject=DWW Skript,
pdfmenubar=true, 
pdfborder={0 0 0},
colorlinks=true, 
pdfpagelayout=SinglePage,
pdffitwindow=true,
linkcolor=black,
citecolor=citecol,
urlcolor=blue 
}

%% Mehrzeilige Gleichungen
\usepackage{IEEEtrantools}

%% Schoene Tabellen
\usepackage{booktabs}

