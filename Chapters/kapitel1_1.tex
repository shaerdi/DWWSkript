\subsection{Klassifikation}
%\begin{IEEEeqnarray}{rCll}
    %\left( y'' \right)^{4} &=& 5 & \mbox{\hspace{3em}ODE, Ordnung 2} \\
    %\frac{d^2x}{\left( d t \right)^{2}} + 2
    %\frac{d^2x}{\left( d t \right)^{2}} &=& x\left( t \right) &
    %\mbox{\hspace{3em}ODE, Ordnung 2} \\
    %\frac{\partial f}{\left( \partial x \right)} - 
    %\frac{\partial f}{\left( \partial t \right)} &=& 0 & 
    %\mbox{\hspace{3em}PDE, Ordnung 1}
%\end{IEEEeqnarray}
\subsubsection*{a) elementar Lösbar}
Eine DGL ist elementar Lösbar wenn sie als 
\begin{equation*}
    y^{\left( n \right)} = f(x)
\end{equation*}
geschrieben werden kann.

\subsubsection*{b) Ordnung}
\definition{DGL n-ter Ordnung}{Hat Ableitungen bis zur Ordnung n}
%\begin{outline}
    %\1 \eqref{eq:exampleKlass:1} hat eine Ableitung der Ordnung~2 $\Rightarrow$
       %die DGL hat ebenfalls Ordnung~2.
    %\1 \eqref{eq:exampleKlass:2} hat eine Ableitung der Ordnung~1 und eine
       %Ableitung der Ordnung~2 $\Rightarrow$ die DGL hat Ordnung~2.
    %\1 \eqref{eq:exampleKlass:3} hat eine Ableitungen bis zu der Ordnung~1 
       %$\Rightarrow$ die DGL hat Ordnung~1.
%\end{outline}
\bemerkung{DGL n-ter Ordngung hat als allg. Lösung i.A. eine Funktion mit n
Parametern.}

\textbf{Spezialfall:} Eine Differentialgleichung 1. Ordnung, die in der Form 
\begin{equation*}
    y' = \frac{g(x)}{f(y)}
\end{equation*}
geschrieben werden kann, heisst \underline{separierbar}.

\subsubsection*{c) Linearität}
Eine Differentialgleichung ist linear, wenn die gesuchte Funktion und alle
Ab\-lei\-tun\-gen nur mit der Potenz 1 vorkommen und wenn keine Mischterme vorhanden
sind.

Lineare Differentialgleichungen können immer in die Form 
\begin{equation*}
    g(x) = a(x)\cdot y + b(x) \cdot y' + c(x)\cdot y'' + \ldots
\end{equation*}
gebracht werden.

\textbf{Spezialfall:} Eine lineare Differentialgleichung ist
\underline{homogen}, wenn der Anregungsterm $g(x)=0$ 
(D.h. die Funktion oder deren Ableitungen in allen Termen vorkommen).

\textbf{Spezialfall:} Eine lineare Differentialgleichung hat \underline{%
konstante Koeffizienten}, wenn alle Koeffizienten $a,b,c\ldots$ nicht von der
Funktionsvariablen abhängen.
%\begin{outline}
    %\1 In \eqref{eq:exampleKlass:1} ist $x$ die Funkionsvariable, und kommt
       %nicht vor $\Rightarrow$ die DGL ist homogen.
    %\1 In \eqref{eq:exampleKlass:2} ist $t$ die Funkionsvariable, und kommt
       %nicht vor $\Rightarrow$ die DGL ist homogen.
    %\1 \eqref{eq:exampleKlass:3} hat hat $x$ und $t$ als Funkionsvariablen,
    %der Anregungsterm $t^{2}$ macht die DGL inhomogen.  
%\end{outline}

\subsubsection*{d) Ableitungen}
\definition{Gewöhnliche DGL (ODE)}{Hat nur Ableitungen nach einer Variablen.}
\definition{Partielle DGL (PDE)}{Hat Ableitungen nach mehreren Variablen.}
%\begin{outline}
    %\1 \eqref{eq:exampleKlass:1} hat nur Ableitungen nach $x$ $\Rightarrow$
       %die DGL ist gewöhnlich
    %\1 \eqref{eq:exampleKlass:2} hat nur Ableitungen nach $t$ $\Rightarrow$
       %die DGL ist gewöhnlich
    %\1 \eqref{eq:exampleKlass:3} hat Ableitungen nach $x$ und $t$ $\Rightarrow$
       %es handelt sich um eine partielle DGL.
%\end{outline}

\bsp{Beispiele}
\begin{equation*}
    \left( y'' \right)^{4} = 5y% \label{eq:exampleKlass:1}\\
\end{equation*}

\begin{outline}
    \1 Es ist nicht möglich, die Gleichung als $y''=f(x)$ zu schreiben, sie
    ist nicht elementar lösbar.
    \1 Die maximale auftretende Ableitung ist $y''\Rightarrow$ Ordnung 2.
    \1 Die Ableitung $y''$ hat die Potenz 4 $\Rightarrow$ nichtlinear.
    \1 nicht-linear $\Rightarrow$ nicht auf Homogenität prüfen.
    \1 Es wird nur nach $x$ abgeleitet $\Rightarrow$ ODE.
\end{outline}

\begin{equation*}
    \frac{d^2x}{\left( d t \right)^{2}} + 2
    \frac{d x}{\left( d t \right)} = x\left( t \right)
\end{equation*}

\begin{outline}
    \1 Es ist nicht möglich, die Gleichung als $\ddot{x}=f(t)$ zu schreiben, sie
    ist nicht elementar lösbar.
    \1 Die maximale auftretende Ableitung ist $\ddot{x}\Rightarrow$ Ordnung 2.
    \1 Alle Ableitungen von $x$ kommen nur mit der Potenz 1 vor $\Rightarrow$ linear.
    \1 Die Funktion $x$ oder Ableitungen davon kommen in allen Termen vor $\Rightarrow$ Homogen.
    \1 Es wird nur nach $t$ abgeleitet $\Rightarrow$ ODE.
\end{outline}

\begin{equation*}
    \frac{\partial f}{\left( \partial x \right)} - 
    \frac{\partial f}{\left( \partial t \right)} = t^{2} 
\end{equation*}

\begin{outline}
    \1 Es ist nicht möglich, die Gleichung als $f=g(t,x)$ zu schreiben, sie
    ist nicht elementar lösbar.
    \1 Die maximale auftretende Ableitung ist $f'$ bzw $\dot{f}\Rightarrow$ Ordnung 1.
    \1 Alle Ableitungen von $f$ kommen nur mit der Potenz 1 vor $\Rightarrow$ linear.
    \1 Die Funktion $f$ kommt nicht in allen Termen vor $\Rightarrow$ inhomogen.\\
       (Anregungsterm $g(t)=t^{2}$)
    \1 Es wird nach $t$ und nach $x$ abgeleitet $\Rightarrow$ PDE.
\end{outline}
%\begin{IEEEeqnarray*}{rCl}
    %\left( y'' \right)^{4} &=& 5\\% \label{eq:exampleKlass:1}\\
    %\frac{d^2x}{\left( d t \right)^{2}} + 2
    %\frac{d x}{\left( d t \right)} &=& x\left( t \right)\\
    %%\label{eq:exampleKlass:2} \\
    %\frac{\partial f}{\left( \partial x \right)} - 
    %\frac{\partial f}{\left( \partial t \right)} &=& t^{2} 
    %%\label{eq:exampleKlass:3}
%\end{IEEEeqnarray*}



