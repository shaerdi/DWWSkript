\subsection{Klassifikation}
\subsubsection*{a) elementar Lösbar}
Eine DGL ist elementar Lösbar, wenn sie als 
\begin{equation*}
    y^{\left( n \right)} = f(x)
\end{equation*}
geschrieben werden kann.

\subsubsection*{b) Ordnung}
\definition{DGL n-ter Ordnung}{Hat Ableitungen bis zur Ordnung $n$.}
\bemerkung{DGL n-ter Ordnung hat als allg. Lösung i.A. eine Funktion mit n
Parametern.}

\textbf{Spezialfall:} Eine Differentialgleichung 1. Ordnung, die in der Form 
\begin{equation*}
    y' = \frac{g(x)}{f(y)}
\end{equation*}
geschrieben werden kann, heisst \underline{separierbar}.

\subsubsection*{c) Linearität}
Eine Differentialgleichung ist linear, wenn die gesuchte Funktion und alle
Ab\-lei\-tun\-gen nur mit der Potenz 1 vorkommen und wenn keine Mischterme vorhanden
sind.

Lineare Differentialgleichungen können immer in die Form 
\begin{equation*}
    g(x) = a(x)\cdot y + b(x) \cdot y' + c(x)\cdot y'' + \ldots
\end{equation*}
gebracht werden.

\textbf{Spezialfall:} Eine lineare Differentialgleichung ist
\underline{homogen}, wenn der Anregungsterm $g(x)=0$ \\
(Die Funktion oder deren Ableitungen kommt in allen Summentermen vor).

\textbf{Spezialfall:} Eine lineare Differentialgleichung hat \underline{%
konstante Koeffizienten}, wenn alle Koeffizienten $a,b,c\ldots$ nicht von der
Funktionsvariablen abhängen.

\bsp{Beispiele}
\begin{equation*}
    \left( y'' \right)^{4} = 5y% \label{eq:exampleKlass:1}\\
\end{equation*}

\begin{outline}
    \1[a)] Es ist nicht möglich, die Gleichung als $y''=f(x)$ zu schreiben, sie
           ist nicht elementar lösbar.
    \1[b)] Die maximale auftretende Ableitung ist $y''\Rightarrow$ Ordnung 2.
        \2 Ordnung 2 $\Rightarrow$ nicht auf Separierbarkeit prüfen.
    \1[c)] Die Ableitung $y''$ hat die Potenz 4 $\Rightarrow$ nichtlinear.
        \2 nicht-linear $\Rightarrow$ nicht auf Homogenität oder konstante
           Koeffizienten prüfen.
\end{outline}

\begin{equation*}
    \frac{\text{d}^2x}{ \text{d} t ^{2}} + 2
    \frac{\text{d} x}{ \text{d} t } = x\left( t \right)
\end{equation*}

\begin{outline}
    \1[a)] Es ist nicht möglich, die Gleichung als $\ddot{x}=f(t)$ zu 
           schreiben, sie ist nicht elementar lösbar.
    \1[b)] Die maximale auftretende Ableitung ist $\ddot{x}\Rightarrow$ 
           Ordnung~2.
        \2 Ordnung~2 $\Rightarrow$ nicht auf Separierbarkeit prüfen.
    \1[c)] Alle Ableitungen von $x$ kommen nur mit der Potenz~1 vor
           $\Rightarrow$ linear.
        \2 Die Funktion $x$ oder Ableitungen davon kommt in allen Termen 
           vor $\Rightarrow$ Homogen.
        \2 Die Funktionsvariable $t$ kommt nicht in den Koeffizienten vor
           $\Rightarrow$ konstante Koeffizienten.
\end{outline}

\begin{equation*}
    3y^{(1)} y= \frac{x}{y}
\end{equation*}

\begin{outline}
    \1[a)] Es ist nicht möglich, die Gleichung als $y=f(x)$ zu schreiben, sie
       ist nicht elementar lösbar.
    \1[b)] Die maximale auftretende Ableitung ist $y^{(1)} \Rightarrow$ 
           Ordnung~1.
        \2 Die Funktion lässt sich als $y' = \frac{x}{3y^2}=\frac{g(x)}{f(y}$
            schreiben $\Rightarrow$ separierbar.
    \1[c)] $y^2 \Rightarrow$ nicht-linear.
        \2 nicht-linear $\Rightarrow$ nicht auf Homogenität oder konstante
           Koeffizienten prüfen.
\end{outline}
