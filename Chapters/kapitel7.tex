\section{Numerisches Lösen von DGL}

\subsection{Das Euler Verfahren}
DGL 1. Ordnung mit Anfangswert:
\begin{equation*}
    \tikzmark{a}\dot{x} = f(\tikzmark{b}t,x), \hspace{1em} x(0)=x_0
\end{equation*}
\begin{center}
    \begin{tikzpicture}[overlay,remember picture]
        \node at (-4,0) (ta) {Steigung};
        \node[right] at (-2,0) (tb) {Entspricht Funktion $f$ der gesuchten Variable $x$};
        \draw[->] (ta) [in=-90]to (a);
        \draw[->] (tb) [in=-90,out=100]to (b);
    \end{tikzpicture}
\end{center}
Idee: Immer nur ein Schritt auf einmal.
\begin{center}
    \begin{tikzpicture}
        \draw (-1,0) -- (0,0) -- (0,-1);
        \draw[<->] (6,0) node[below] {$t$} -- (0,0) -- (0,3) node[left] {$x$};
        \draw[thick] (0,1) node[left] {$x_0$} -- ++ (3,0) -- ++ (0,1)
        node[midway, right] {$\Delta x$}   -- (0,1);
        \draw (3,-0.1) node[below] {$t_1$} -- (3,0.1);
        \draw (0.1,2)  -- (-0.1,2) node[left] {$x_1$};
        \node[below] at (1.5,0) {$\Delta t = h$};
        \node at ( 7,2) {$\dfrac{\Delta x}{h} = f(t,x_0)$};
    \end{tikzpicture}
\end{center}
Explizites Euler Verfahren:
\begin{equation*}
    \boxed{
        x_{n+1} = x_n + h\cdot f\left( t_n,x_n \right)
    }
\end{equation*}
Implizites Euler Verfahren:
\begin{equation*}
    \boxed{
        x_{n+1} = x_n + h\cdot f\left( t+h,x_{n+1} \right)
    }
\end{equation*}

\bsp{Beispiel:}




\subsection{Runge-Kutta Verfahren}
\subsection{Differentialgleichungssysteme}
\subsection{Matlab ode}
