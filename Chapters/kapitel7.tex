\section{Numerisches Lösen von DGL}

\subsection{Das Euler Verfahren}
DGL 1. Ordnung mit Anfangswert:
\begin{equation*}
    \tikzmark{a}\dot{x} = f(\tikzmark{b}t,x), \hspace{1em} x(0)=x_0
\end{equation*}
\begin{center}
    \begin{tikzpicture}[overlay,remember picture]
        \node at (-4,0) (ta) {Steigung};
        \node[right] at (-2,0) (tb) {Entspricht Funktion $f$ der gesuchten Variable $x$};
        \draw[->] (ta) [in=-90]to (a);
        \draw[->] (tb) [in=-90,out=100]to (b);
    \end{tikzpicture}
\end{center}
Idee: Immer nur ein Schritt auf einmal.
\begin{center}
    \begin{tikzpicture}
        \draw (-1,0) -- (0,0) -- (0,-1);
        \draw[<->] (6,0) node[below] {$t$} -- (0,0) -- (0,3) node[left] {$x$};
        \draw[thick] (0,1) node[left] {$x_0$} -- ++ (3,0) -- ++ (0,1)
        node[midway, right] {$\Delta x$}   -- (0,1);
        \draw (3,-0.1) node[below] {$t_1$} -- (3,0.1);
        \draw (0.1,2)  -- (-0.1,2) node[left] {$x_1$};
        \node[below] at (1.5,0) {$\Delta t = h$};
        \node at ( 7,2) {$\dfrac{\Delta x}{h} = f(t,x_0)$};
    \end{tikzpicture}
\end{center}
Explizites Euler Verfahren:
\begin{equation*}
    \boxed{
        x_{n+1} = x_n + h\cdot f\left( t_n,x_n \right)
    }
\end{equation*}
Implizites Euler Verfahren:
\begin{equation*}
    \boxed{
        x_{n+1} = x_n + h\cdot f\left( t+h,x_{n+1} \right)
    }
\end{equation*}

\bsp{Beispiel:}
Gegeben ist die DGL
\begin{equation*}
    \dot{x} = x^2+2t\cdot x, \hspace{1em} x_0 = 2
\end{equation*}
Nicht linear und nicht separierbar $\Rightarrow$ numerisch lösen.

Expliziter Euler mit Schrittweite $h=1$:
\begin{eqnarr}
    \dot{x} &=&  f(t,x) \\
    &=&  x^2+2t\cdot x\\
    &\Rightarrow&\\
    x_{n+1} &=& x_n + h\cdot \left( x_n^2+2t_n\cdot x_n \right)\\
    x_{1} &=& x_0 + h\cdot \left( x_0^2+2t_0\cdot x_0 \right)\\
    &=& 2 + 1\cdot \left( 2^2+2\cdot 0 \cdot 2 \right)\\
    &=& 6\\
    x_2 &=&  x_1 + h\cdot \left( x_1^2+2t_1\cdot x_1 \right)\\
    &=& 6 + 1\cdot\left( 6^2+2\cdot 1\cdot 6 \right)\\
    &=& 54\\
    \text{usw.}&&\\
\end{eqnarr}

\subsection{Runge-Kutta Verfahren}
\subsubsection{Runge-Kutta 2}
(Mittelpunktsregel)
\begin{eqnarr}
    x_{n+\frac{1}{2}} &=&  x_n + \frac{h}{2}\cdot f\left( t_n,x_{n} \right)\\
    x_{n+1} &=& x_n + h\cdot f\left(t_n+\frac{h}{2},x_{n+\frac{1}{2}}\right)\\
\end{eqnarr}
\begin{center}
    \begin{tikzpicture}
    \begin{axis}[
        width = 10cm,
        height = 5cm,
        axis lines = middle,
        clip = false,
        xmin = 0,
        ymin = 0,
        ymax = 4, 
        restrict y to domain=0:4,
        ticks=none,
        xlabel = $t$,
        ylabel = $x$,
    ]

    \addplot+[mark=none,samples=150,domain=0:6 ] {x^2/4};

    \draw (axis cs:2,0.1) -- ++(axis cs:0,-0.2) node[below] {$t_{n+\frac{1}{2}}$};
    \draw (axis cs:4,0.1) -- ++(axis cs:0,-0.2) node[below] {$t_n$};
    
    %\draw[dotted] (omegaR0) node[below] {$\omega_0$} -- ++ (axis cs:0, 0.2);
    %\draw[dotted] (omegaR1) node[below] {$\underset{\delta=1}{\omega_r}$} 
        %-- ++ (axis cs:0, {A(wr(1),1});
    %\draw[dotted] (omegaR2) node[below] {$\underset{\delta=2}{\omega_r}$} 
        %-- ++ (axis cs:0, {A(wr(2),2});
    
\end{axis}
        %\draw (2,0.1) -- ++(0,-0.2) node[below] {$t_{n+\frac{1}{2}}$};
        %\draw (4,0.1) -- ++(0,-0.2) node[below] {$t_n$};
    \end{tikzpicture}
\end{center}
\subsection{Differentialgleichungssysteme}
\subsection{Matlab ode}
