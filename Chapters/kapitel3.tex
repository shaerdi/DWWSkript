\section{Lineare DGL 1. Ordnung}
Repetition:
\begin{equation*}
    y' +f(x)\cdot y = \underbrace{g(x)}_{\mbox{Störfunktion}}
\end{equation*}
Störfunktion $=0$: homogen \\
Störfunktion $\neq0$: inhomogen
$f(x)=$ konstant: Mit konstanten Koeffizienten

\subsection{Hauptsatz}
\begin{center}
    Allgemeine Lösung der linearen DGL\\
    =\\
    allgemeine Lösung der zugehörigen homogenen DGL\\
    +\\
    \underline{irgendeine} partikuläre Lösung der inhomogenen DGL
\end{center}

\textbf{Beweis:}

$y_h$ ist allgemeine Lösung von $y'+f(x)\cdot y=0$ (1 Parameter)\\
$y_p$ ist partikuläre Lösung von $y'+f(x)\cdot y=g(x)$ (0 Parameter)
    \begin{eqnarr}
        &\Rightarrow& \\
        y_p' +f(x)y_p &=& g(x)\\
        y_h' +f(x)y_h &=& 0 \\ 
        &\Rightarrow& \mbox{(Summe)}\\
        y_h' + y_p' + f(x) y_h + f(x) y_p &=& g(x)\\
        y_h' + y_p' + f(x) \left( y_h + y_p \right) &=& g(x)\\
        \left(y_h + y_p\right)' + f(x) \left( y_h + y_p \right) &=& g(x)
    \end{eqnarr}
$\Rightarrow y_h+y_p$ ist allgemeine (weil 1 Parameter) Lösung von
$y'+f(x)y=g(x)$

\bsp{Beispiel:}
\begin{equation*}
    y'+y = 2e^{x}
\end{equation*}

homogen:
\begin{equation*}
    y'+y=0
\end{equation*}
Separierbar:
\begin{eqnarr}
    \frac{\dd{y}}{\dd{x}} &=&  -y \\
    \int \frac{1}{y}\dd{y} &=& \int -1 \dd{x}\\
    \ln(y) &=&  -x + C_1\\
    y_h &=& C\cdot e^{-x} \hspace{3em}\mbox{(allgemeine Lösung)}
\end{eqnarr}

inhomogen:
\begin{equation*}
    y'+y=2\cdot e^{x}
\end{equation*}
Raten: $y_p\left( x \right)=e^{x}$\\
$\Rightarrow y(x)=y_h+h_p=Ce^{-x}+e^x$

\subsection{Allgemeine Lösung der homogenen DGL}
Homogene DGL:
\begin{equation*}
    y' + f(x) \cdot y = 0
\end{equation*}

Lineare, homogene DGL 1. Ordnung ist immer separierbar $\Rightarrow$
\begin{eqnarr}
    y' + f(x) \cdot y &=&  0\\
    y' &=&  -f(x)\cdot y\\
    \frac{\dd{y}}{\dd{x}} &=&  -f(x)\cdot y\\
    \frac{1}{y}\dd{y}&=&  -f(x)\dd{x} \\
    \int\frac{1}{y}\dd{y}&=& \int -f(x)\dd{x} \\
    \ln\left( y \right)&=&  -F(x) + C_1 \\
    y&=&  e^{-F(x) + C_1} \\
    &=&  C\cdot e^{-F(x)} \\
\end{eqnarr}

\subsection{Partikuläre Lösung der inhomoegenen DGL}
Methode
\begin{outline}
    \1 Raten (z.B. mit Richtungsfeld) und Einsetzen
    \1 Variation der Konstanten (Kapitel \ref{variationdK})
    \1 Spezialfälle: Tabellen (Papula)
\end{outline}

\subsection{Variation der Konstanten} \label{variationdK}
Die Lösung der homogenen DGL hat immer die Form
\begin{equation*}
    y_h\left( x \right) = C\cdot e^{-F(x)}
\end{equation*}
Die Lösung der inhomogenen DGL \emph{sollte} so ähnlich aussehen.

Idee: ersetze Konstante C durch die Funktion $C\left( x \right)$ und setze sie in die DGL ein. Dadurch kann $C\left( x \right)$ berechnet, und $y_p$ bestimmt werden.

\bsp{Beispiel}
\begin{equation*}
    y'-\frac{y}{x} = x^{2}
\end{equation*}

Homogene Lösung $y_h$:
\begin{eqnarr}
    y' \underbrace{- \frac{1}{x}}_{f(x)}y &=& 0\\
    &\Rightarrow& \\ 
    y_h &=&  C\cdot e^{-F(x)} \\
    &=& C\cdot e^{--\ln(x)} \\
    &=& C\cdot x
\end{eqnarr}

Partikuläre Lösung $y_p$:

Ansatz: $y_p=C(x)\cdot x$ in \underline{inhomogene} DGL einsetzen:
\begin{eqnarr}
    \left[ C\left( x \right)\cdot x \right]' - \frac{\left[ C(x)\cdot x \right]}{x} &=& x^2 \\
    C\left( x \right)'\cdot x + C(x)\cdot 1 - C(x)&=& x^2 \\
    C\left( x \right)'\cdot x &=& x^2 \\
    C\left( x \right)'&=& x \\
    C\left( x \right)&=& \frac{1}{2}x^2 \\
\end{eqnarr}
$\Rightarrow$
\begin{equation*}
    y_p(x) = C(x)\cdot x = \frac{1}{2}x^2\cdot x = \frac{x^3}{2}
\end{equation*}
$\Rightarrow$
\begin{equation*}
    y(x) = y_h+y_p= C\cdot x+ \frac{x^3}{2}
\end{equation*}

\bemerkung{Die Variation der Konstanten führt \underline{immer} zu einer elementar lösbaren DGL für $C(x)'$. Der Term $C(x)$ kürzt sich jeweils weg.} 

\subsection{Spezialfall konstante Koeffizienten}
Für 
\begin{equation*}
    y'+\underbrace{a}_{\mbox{konst. Koeff.}}\cdot y = g(x)
\end{equation*}
Dies ist ein einfacher Spezialfall:
\begin{outline}
    \1 Die homogene Lösung ist einfacher
    \1 Für die partikuläre Lösung muss man nicht die Konstante variieren, sondern man kann (in vielen Fällen) in einer Tabelle nachschlagen
\end{outline}
Homoegen Lösung:
\begin{equation*}
    y_h = C\cdot e^{-F(x)} = C\cdot e^{-ax}
\end{equation*}

\bsp{Beispiel}
\begin{equation*}
    y'\underbrace{-2}_{a=-2} \cdot y = \underbrace{4x -2}_{g(x)}
\end{equation*}
Homogen:
\begin{equation*}
    y_h = C\cdot e^{--2x} = C\cdot e^{2x}
\end{equation*}
Inhomogen: Ansatz aus Tabelle $y_p=c_1\cdot x+c_0$
\begin{eqnarr}
    \left( c_1\cdot x + c_0 \right)'-2\cdot \left( c_1\cdot x+c_0 \right)&=& 4x-2\\
    c_1 - 2\cdot c_1\cdot x-2\cdot c_0 &=& 4x-2\\
    &\Rightarrow& \\
    -2\cdot c_1 &=& 4\\
    c_1 - 2\cdot c_0 &=& -2 \\
    &\Rightarrow& \\
    c_1 &=& -2\\
    c_0 &=& 0\\
    &\Rightarrow& \\
    y_p &=&  -2\cdot x
\end{eqnarr}
somit $y(x) = y_h+y_p=C\cdot e^{2x}-2x$

