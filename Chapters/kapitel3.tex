\section{Lineare DGL 1. Ordnung}
Repetition:
\begin{equation*}
    y' +f(x)\cdot y = \underbrace{g(x)}_{\mbox{Störfunktion}}
\end{equation*}
Störfunktion $=0$: homogen \\
Störfunktion $\neq0$: inhomogen
$f(x)=$ konstant: Mit konstanten Koeffizienten

\subsection{Hauptsatz}
\begin{center}
    Allgemeine Lösung der linearen DGL\\
    =\\
    allgemeine Lösung der zugehörigen homogenen DGL\\
    +\\
    \underline{irgendeine} partikuläre Lösung der inhomogenen DGL
\end{center}

\textbf{Beweis:}

$y_h$ ist allgemeine Lösung von $y'+f(x)\cdot y=0$ (1 Parameter)\\
$y_p$ ist partikuläre Lösung von $y'+f(x)\cdot y=g(x)$ (0 Parameter)
    \begin{IEEEeqnarray*}{rCl}
        &\Rightarrow& \\
        y_p' +f(x)y_p &=& g(x)\\
        y_h' +f(x)y_h &=& 0 \\ 
        &\Rightarrow& \mbox{(Summe)}\\
        y_h' + y_p' + f(x) y_h + f(x) y_p &=& g(x)\\
        y_h' + y_p' + f(x) \left( y_h + y_p \right) &=& g(x)\\
        \left(y_h + y_p\right)' + f(x) \left( y_h + y_p \right) &=& g(x)
    \end{IEEEeqnarray*}
$\Rightarrow y_h+y_p$ ist allgemeine (weil 1 Parameter) Lösung von
$y'+f(x)y=g(x)$

\bsp{Beispiel:}
\begin{equation*}
    y'+y = 2e^{x}
\end{equation*}

homogen:
\begin{equation*}
    y'+y=0
\end{equation*}
Separierbar:
\begin{IEEEeqnarray*}{rCl}
    \frac{\dd{y}}{\dd{x}} &=&  -y \\
    \int \frac{1}{y}\dd{y} &=& \int -1 \dd{x}\\
    \ln(y) &=&  -x + C_1\\
    y_h &=& C\cdot e^{-x} \hspace{3em}\mbox{(allgemeine Lösung)}
\end{IEEEeqnarray*}

inhomogen:
\begin{equation*}
    y'+y=2\cdot e^{x}
\end{equation*}
Raten: $y_p\left( x \right)=e^{x}$\\
$\Rightarrow y(x)=y_h+h_p=Ce^{-x}+e^x$

\subsection{Allgemeine Lösung der homogenen DGL}


