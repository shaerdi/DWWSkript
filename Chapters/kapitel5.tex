\section{Schwingungen}
\subsection{Freie Schwingung}
\bsp{Beispiel}

\begin{minipage}{0.47\textwidth}
    Feder~-~Dämpfer~-~System.
    
    Eine Masse $m$ hängt an einer Feder und ist
    an einen Flüssig-Dämpfer angeschlossen.
\end{minipage}
\begin{minipage}[r]{0.49\textwidth}
    \hfill
    \begin{circuitikz}
        \draw (-1,0) -- (1,0);
        \foreach \x in {-9,...,9}
            \draw ( {(\x-1)/10}, 0.1 ) -- ({\x/10},0);
        \coordinate (mass) at (0,-1.7);
        \draw (0,0) to [L] (mass) ;
        \node at (0,-1) [right,xshift=0.2cm] {Federkonstante $k$} ;
        \fill (mass) circle [radius=0.2] node[left,xshift=-0.1cm] {$m$};
        \draw (mass) -- (0,-3) -- + (0.5,0) -- + (-0.5,0);
        \draw (-1,-2.5) -- ++ (0,-1) -- ++ (2,0) -- ++ (0,1);
        \draw[decorate,decoration=snake] (-1, -2.7) -- ++ (2,0);
        \node[right] at (1,-3) {Dämpfer $r$};
        \draw[->] (-1.5,-3) -- ++ (0,2.5) node[midway,left] {$x$};
    \end{circuitikz}
\end{minipage}

DGL: $F=a\cdot m = -k\cdot x-r\cdot \dot{x}$
\begin{equation*}
\boxed{\ddot{x} + \frac{r}{m}\cdot\dot{x}+\frac{k}{m}\cdot x=0}
\end{equation*}
Freie, gedämpfte Schwingung

Lösung:
\begin{equation*}
    \ddot{x} + \underbrace{\frac{r}{m}}_{2\delta}\cdot\dot{x}+
    \underbrace{\frac{k}{m}}_{\omega_0^2}\cdot x=0
\end{equation*}
\begin{eqnarr}
    \delta^2 - \omega_0^2 &=&  \left( \frac{r}{2m} \right)^2-\frac{k}{m}\\
    &>& 0 \Rightarrow \text{Fall 1 (Kriechfall)} \\
    &=& 0 \Rightarrow \text{Fall 2 (aperiodischer Grenzfall)} \\
    &<& 0 \Rightarrow \text{Fall 3 (Schwingungsfall)} \\
\end{eqnarr}

\subsection*{a) Starke Dämpfung (Kriechfall)}
\begin{equation*}
    \boxed{\delta>\omega_0}
\end{equation*}
\begin{equation*}
    \lambda_{1,2} = -\delta\pm\underbrace{\sqrt{\delta^2-\omega_0^2}}_
    {>0 \text{ und }<\delta}<0
\end{equation*}
Lösung: \begin{equation*}
    x(t) = C_1\cdot e^{\lambda_1\cdot t} + C_2\cdot e^{\lambda_2\cdot t}
\end{equation*}

\subsection*{b) Aperiodischer Grenzfall}
\begin{equation*}
    \boxed{\delta=\omega_0}
\end{equation*}
\begin{equation*}
    \lambda = -\delta
\end{equation*}
Lösung: \begin{equation*}
    x(t) = C_1\cdot e^{-\delta\cdot t} + C_2\cdot t\cdot e^{-\delta\cdot t}
\end{equation*}

\subsection*{c) Schwache Dämpfung}
\begin{equation*}
    \boxed{\delta<\omega_0}
\end{equation*}
\begin{equation*}
    \omega = \sqrt{\omega_0^2-\delta^2}<\omega_0
\end{equation*}
Lösung: \begin{eqnarr}
    x(t)&=&  e^{-\delta \cdot t}\cdot \left[
        C_1\cdot\sin(\omega_d \cdot t) +
        C_2\cdot\cos(\omega_d \cdot t) 
        \right]\\
        &=& C\cdot e^{-\delta \cdot t} \cdot
        \sin\left( \omega_d\cdot t+\phi_d \right)\\
        &=& C\cdot e^{-\delta \cdot t} \cdot
        \cos\left( \omega_d\cdot t+\varphi_d \right)
\end{eqnarr}
