\section{Lineare DGL 2. Ordnung}
(mit konstanten Koeffizienten)

Allgemeine Form einer lin. DGL 2. Ordnung mit konst. Koeff:
\begin{equation*}
    y'' + a \cdot y' + b\cdot y = g(x)
\end{equation*}

\textbf{Satz}
\begin{equation*}
    y(x) = y_h(x) + y_p(x)
\end{equation*}
Analog zur linearen DGL 1. Ordnung, Lösung in zwei Schritten:
\begin{outline}
    \1 $y_h$ finden: Formel aus \ref{algloeshomdgl}
    \1 $y_p$ finden: Tabelle verwenden (\ref{aufsuchenpartloes})
\end{outline}

\subsection{Allgemeine Eigenschaften der homogenen DGL}
\begin{equation*}
    y'' + a\cdot y' + b\cdot y = 0 \hspace{2em} (*)
\end{equation*}

\underline{Satz}
\begin{outline}    
    \1[\circled{1}] Ist $y(x)$ eine Lösung von (*), dann auch $C\cdot y(x),C\in\mathrm{R}$
    \1[\circled{2}] Sind $y_1(x)$ und $y_2(x)$ Lösngen von (*), dann auch $y_1(x)+y_2(x)$
    \1[\circled{3}] Ist $y(x)=u(x)+j\cdot v(x)$ eine \emph{komplexe} Lösung von (*), dann auch $u(x)$ und $v(x)$\end{outline}

\bsp{Beispiel}
\begin{equation*}
    y'' + \omega ^2 \cdot y =0, \hspace{2em} (\omega >0)
\end{equation*}

$y = \sin(\omega x)$ ist Lösung:
\begin{eqnarr}
    \left[ \sin(\omega x) \right]'' + \omega ^2 \left[ \sin(\omega x) \right] &=& 0\\
    -\omega^2\sin(\omega x)  + \omega ^2 \sin(\omega x) &=& 0\\
\end{eqnarr}

$y = \cos(\omega x)$ ist Lösung:
\begin{eqnarr}
    \left[ \sin(\omega x) \right]'' + \omega ^2 \left[ \sin(\omega x) \right] &=& 0\\
    -\omega^2\sin(\omega x)  + \omega ^2 \sin(\omega x) &=& 0\\
\end{eqnarr}

Aus \circled{1} folgt, dass $y_1=C_1\cdot\sin(\omega x)$ und $y_2=C_2\cdot\cos(\omega x)$ ebenfalls Lösungen sind.\\
Aus \circled{2} folgt, dass $y=y_1+y_2$ ebenfalls eine Lösung ist

\begin{equation*}
    \left[ C_1\cdot\sin(\omega x)+C_2\cos(\omega x) \right]'' + \omega^2\cdot\left[ C_1\cdot\sin(\omega x)+C_2\cos(\omega x) \right] =0
\end{equation*}

$\Rightarrow y= C_1\cdot\sin(\omega x)+C_2\cos(\omega x) $ hat 2 Parameter ($C_1$ und $C_2$) und ist deshalb nicht nur eine, sondern die allgemeine Lösung.

\bsp{Def} Zwei Funktionen heissen \underline{linear unabhängig}, falls die eine \underline{nicht} vielfaches der anderen ist.

\bsp{Beispiele}
\begin{outline}
    \1 $\left.\begin{array}{c}y_1=\sin x\\y_2=\cos x\end{array}\right\}$
\end{outline}

\subsection{Allgemeine Lösung der homogenen DGL} \label{algloeshomdgl}
\subsection{Partikuläre Lösung der inhomogenen DGL} \label{aufsuchenpartloes}
