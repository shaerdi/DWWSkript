\section{Lineare DGL 2. Ordnung}
(mit konstanten Koeffizienten)

Allgemeine Form einer lin. DGL 2. Ordnung mit konst. Koeff:
\begin{equation*}
    y'' + a \cdot y' + b\cdot y = g(x)
\end{equation*}

\textbf{Satz}
\begin{equation*}
    y(x) = y_h(x) + y_p(x)
\end{equation*}
Analog zur linearen DGL 1. Ordnung, Lösung in zwei Schritten:
\begin{outline}
    \1 $y_h$ finden: Formel aus \ref{algloeshomdgl}
    \1 $y_p$ finden: Tabelle verwenden (\ref{aufsuchenpartloes})
\end{outline}

\subsection{Allgemeine Eigenschaften der homogenen DGL}
\begin{equation*}
    y'' + a\cdot y' + b\cdot y = 0 \hspace{2em} (*)
\end{equation*}

\underline{Satz}
\begin{outline}    
    \1[\circled{1}] Ist $y(x)$ eine Lösung von (*), dann auch $C\cdot
    y(x),C\in\mathbb{R}$
    \1[\circled{2}] Sind $y_1(x)$ und $y_2(x)$ Lösngen von (*), dann auch $y_1(x)+y_2(x)$
    \1[\circled{3}] Ist $y(x)=u(x)+j\cdot v(x)$ eine \emph{komplexe} Lösung von (*), dann auch $u(x)$ und $v(x)$\end{outline}

\bsp{Beispiel}
\begin{equation*}
    y'' + \omega ^2 \cdot y =0, \hspace{2em} (\omega >0)
\end{equation*}

$y = \sin(\omega x)$ ist Lösung:
\begin{eqnarr}
    \left[ \sin(\omega x) \right]'' + \omega ^2 \left[ \sin(\omega x) \right] &=& 0\\
    -\omega^2\sin(\omega x)  + \omega ^2 \sin(\omega x) &=& 0\\
\end{eqnarr}

$y = \cos(\omega x)$ ist Lösung:
\begin{eqnarr}
    \left[ \sin(\omega x) \right]'' + \omega ^2 \left[ \sin(\omega x) \right] &=& 0\\
    -\omega^2\sin(\omega x)  + \omega ^2 \sin(\omega x) &=& 0\\
\end{eqnarr}

Aus \circled{1} folgt, dass $y_1=C_1\cdot\sin(\omega x)$ und $y_2=C_2\cdot\cos(\omega x)$ ebenfalls Lösungen sind.\\
Aus \circled{2} folgt, dass $y=y_1+y_2$ ebenfalls eine Lösung ist

\begin{equation*}
    \left[ C_1\cdot\sin(\omega x)+C_2\cos(\omega x) \right]'' + \omega^2\cdot\left[ C_1\cdot\sin(\omega x)+C_2\cos(\omega x) \right] =0
\end{equation*}

$\Rightarrow y= C_1\cdot\sin(\omega x)+C_2\cos(\omega x) $ hat 2 Parameter
($C_1$ und $C_2$) und ist deshalb nicht nur eine, sondern die allgemeine Lösung.

Wenn $\sin(\omega x)$ eine Lösung ist, dann sind auch $C_1\sin(\omega x)$
und $C_2\sin(\omega x)$ Lösungen. Ist $y = C_1\sin(\omega x)+
C_2\sin(\omega x)$ die allgemeine Lösung?

\bsp{Def} Zwei Funktionen heissen \underline{linear unabhängig}, falls die eine
\underline{nicht} vielfaches der anderen ist.

\bsp{Beispiele}
\begin{outline}
    \1 $\left.\begin{array}{l}y_1=\sin x\\y_2=\cos x\end{array}\right\}$ linear
        unabhängig
    \1 $\left.\begin{array}{l}y_1=4 x\\y_2=2 x\end{array}\right\}$ linear
        abhängig
    \1 $\left.\begin{array}{l}y_1=\sin^2 x\\y_2=2-2\cdot\cos^2 x\end{array}\right\}$ linear abhängig
\end{outline}

\underline{Satz}

Für $y''+a\cdot y' +b\cdot y =0$:\\
Sind $y_1(x)$ und $y_2(x)$ linear unabhängige, partikuläre Lösungen, dann
ist 
\begin{equation*}
    C_1\cdot y_1(x) +C_2\cdot y_2(x), \hspace{2em} (C_1,C_2\in\mathbb{R})
\end{equation*}
die allgemeine Lösung.

\bsp{Bsp} $y'' - 4y' -5y=0$ hat die Lösungen
\begin{eqnarr}
    y_1(x) &=& e^{5x} \\
    y_2(x) &=& e^{-x} \\
\end{eqnarr}
Die Funktionen $y_1(x)$ und $y_2(x)$ sind linear unabhängig $\Rightarrow$
\begin{equation*}
y(x) = C_1\cdot e^{5x}+C_2\cdot e^{-x}
\end{equation*}
ist die allgemeine Lösung.

\subsection{Allgemeine Lösung der homogenen DGL} \label{algloeshomdgl}
Gegeben:
\begin{equation*}
    y''+a\cdot y' + b\cdot y = 0
\end{equation*}
Gesucht: 2 partikuläre, linear unabhängige Lösungen $y_1$ und $y_2$.

Ansatz: $y=e^{\lambda x}$ ($\lambda$: Parameter $\in\mathbb{C}$)

\begin{eqnarr}
    \left[e^{\lambda x}\right]'' + a \cdot \left[e^{\lambda x}\right]' + 
    b\cdot \left[e^{\lambda x}\right] &=& 0 \\
    \lambda^2e^{\lambda x} + a\cdot \lambda \cdot e^{\lambda x} + 
    b\cdot e^{\lambda x} &=& 0 \\
    e^{\lambda x} \cdot \left( \lambda^2+ a\cdot \lambda + b\right) &=& 0 \\
    \lambda^2+ a\cdot \lambda + b &=&  0\\
\end{eqnarr}
Charakteristische Gleichung der DGL, mit der Lösung
\begin{equation*}
    \lambda_{1,2} = \frac{-a\pm\sqrt{a^2-4b}}{2}
\end{equation*}
Fall 1: $\boxed{a^2-4b>0}$

$\Rightarrow$ 2 reelle Lösungen:
\begin{eqnarr}
    \lambda_1 &=& \frac{-a+\sqrt{a^2-4b}}{2} \\
    \lambda_2 &=& \frac{-a-\sqrt{a^2-4b}}{2} \\
\end{eqnarr}
Zwei Lösungen gemäss Ansatz: 
\begin{eqnarr}
    y_1 &=& e^{\lambda_1 \cdot x}\\
    y_2 &=& e^{\lambda_2 \cdot x}\\
\end{eqnarr}
Allgemeine Lösung (\circled{1} und \circled{2})
\begin{equation*}
    \boxed{y(x)=C_1\cdot e^{\lambda_1 \cdot x}+C_2\cdot e^{\lambda_2 \cdot x}}
\end{equation*}
\bsp{Beispiel}
\begin{equation*}
    y''+\tikzmark{a}2y'-\tikzmark{b}8y=0
\end{equation*}
\begin{center}
    \begin{tikzpicture}[overlay,remember picture]
        \node at (-1,0) (ta) {a=2};
        \node at (1,0) (tb) {b=-8};
        \draw[->] (ta) to (a);
        \draw[->] (tb) to (b);
    \end{tikzpicture}
\end{center}
Charakteristische Gleichung:
\begin{eqnarr}
    \lambda^2 +2\lambda -8 &=& 0\\
    \left( \lambda+4 \right)\left( \lambda-2 \right)&=& 0\\
    \lambda_1=-4, && \lambda_2 =2\\
\end{eqnarr}
\begin{equation*}
    y(x)=C_1\cdot e^{-4x}+C_2\cdot e^{2x}
\end{equation*}

Fall 2: $\boxed{a^2-4b=0}$
\begin{equation*}
    \lambda = \frac{-a \pm \sqrt{a^2-4b}}{2} = -\frac{a}{2}
\end{equation*}
Problem: Nur eine Lösung $y_1(x)=e^{-\frac{a}{2}x}$

\bsp{Behauptung}
$y_2(x)=x\cdot e^{\lambda x}$ ist auch eine Lösung

\bsp{Beweis:}
\begin{eqnarr}
    \left[ x\cdot e^{\lambda x} \right]'' + a\cdot\left[ x\cdot e^{\lambda x}
    \right]'+b\cdot \left[ x\cdot e^{\lambda x} \right] &=& 0\\
    \left(1\cdot e^{\lambda x}+\lambda x\cdot e^{\lambda x} \right)' + a\cdot\left(1\cdot e^{\lambda x}+\lambda x\cdot e^{\lambda x} \right)+b\cdot \left( x\cdot e^{\lambda x} \right) &=& 0\\
    \left(\lambda\cdot e^{\lambda x}+\lambda \cdot e^{\lambda x} +\lambda^2 x \cdot e^{\lambda x} \right) + a\cdot\left(1\cdot e^{\lambda x}+\lambda x\cdot e^{\lambda x} \right)+b\cdot \left( x\cdot e^{\lambda x} \right) &=& 0\\
    \underbrace{\left( 2\lambda + a \right)}_{=0,\lambda=-a/2}\cdot e^{\lambda x} +
    \underbrace{\left( \lambda^2+a\lambda
    +b\right)}_{=0\text{, (Char. Glg)}}\cdot x\cdot e^{\lambda x} &=& 0
\end{eqnarr}
Zwei Lösungen $y_1$ und $y_2\Rightarrow$ Allgemeine Lösung
\begin{equation*}
    \boxed{y(x)=C_1\cdot e^{\lambda x}+C_2\cdot x\cdot e^{\lambda x}}
\end{equation*}

\bsp{Beispiel:}
\begin{equation*}
    y''-\tikzmark{a}8y'+\tikzmark{b}16y=0
\end{equation*}
\begin{center}
    \begin{tikzpicture}[overlay,remember picture]
        \node at (-1,0) (ta) {a=-8};
        \node at (1,0) (tb) {b=16};
        \draw[->] (ta) to (a);
        \draw[->] (tb) to (b);
    \end{tikzpicture}
\end{center}
\begin{eqnarr}
    \lambda_1 &=& \frac{8 + \sqrt{(-8)^2-4\cdot 16}}{2}\\
    \lambda_2 &=& \frac{8 - \sqrt{(-8)^2-4\cdot 16}}{2}\\
    \lambda_1 = \lambda_2&=& 4 =\lambda
\end{eqnarr}
$\Rightarrow$ Allgemeine Lösung:
\begin{equation*}
    y(x)=C_1\cdot e^{4x} + C_2\cdot x\cdot e^{4x}
\end{equation*}

Fall 3: $\boxed{a^2-4b<0}$
\begin{eqnarr}
    \lambda_{1,2} &=&  \frac{-a\pm\sqrt{a^2-4b}}{2}\\
    &=&
    \underbrace{\frac{-a}{2}}_{\alpha}\pm
    \sqrt{
        \vphantom{\frac{a^2}{4}}
        \smash{\underbrace{\frac{a^2}{4}-b}_{-\omega^2}}
    }
\end{eqnarr}
Zwei komplexe Lösungen
\begin{equation*}
    \lambda_{1,2} = \alpha\pm j\cdot\omega
\end{equation*}
\begin{equation*}
    \boxed{\alpha=-\frac{a}{2}} \hspace{2em} 
    \boxed{\omega=\sqrt{b-\frac{a^2}{4}}}
\end{equation*}
\begin{equation*}
    \tilde{y}_1(x)=e^{\lambda_1\cdot x},\hspace{2em} 
    \tilde{y}_2(x) = e^{\lambda_2 \cdot x}
\end{equation*}
Das sind zwei ireelle Lösungen. Die Kombination $y=C_1\cdot
\tilde{y}_1+C_2\cdot\tilde{y}_2$ ist die allgemeine Lösung der DGL. Ist es
möglich, die allgemeine Lösung auch als reelle Lösung zu schreiben?
\begin{eqnarr}
    y_1(x) &=&  e^{\left( \alpha+j\omega \right)\cdot x} \\
    &=& e^{\alpha x}\cdot e^{j\omega x}\\
    &=& e^{\alpha x}\left( \cos(\omega x)+j\cdot\sin(\omega x) \right)\\
    \text{Re}\left[ \tilde{y}_1(x) \right] &=& e^{\alpha x}\cdot \cos(\omega x) \\
    \text{Im}\left[ \tilde{y}_1(x) \right] &=& e^{\alpha x}\cdot \sin(\omega x) \\
\end{eqnarr}
$\Rightarrow$ Aus \circled{3} folgt, dass $ y_1=\text{Re}\left[ \tilde{y}_1(x)
\right]$ und $ y_2=\text{Im}\left[ \tilde{y}_1(x) \right]$ auch Lösungen der 
DGL sind. $\Rightarrow$

Allgemeine Lösung:
\begin{equation*}
    \boxed{
        y(x) = C_1 \cdot e^{\alpha x}\cdot \cos(\omega x) + 
             = C_2 \cdot e^{\alpha x}\cdot \sin(\omega x) + 
    }
\end{equation*}

\bsp{Beispiel}
\begin{equation*}
    y''-\tikzmark{a}4y'+\tikzmark{b}13y=0
\end{equation*}
\begin{center}
    \begin{tikzpicture}[overlay,remember picture]
        \node at (-1,0) (ta) {a=4};
        \node at (1,0) (tb) {b=13};
        \draw[->] (ta) to (a);
        \draw[->] (tb) to (b);
    \end{tikzpicture}
\end{center}
\begin{eqnarr}
    a^2 - 4b &=& 16-52\\
    &=& -36\\
    &<& 0\Rightarrow \text{ Fall 3}\\
\end{eqnarr}
\begin{eqnarr}
    \alpha &=& -\frac{a}{2}=-2\\
    \omega &=& \sqrt{b-\frac{a^2}{4}}\\
    &=& \sqrt{13-\frac{16}{4}} = 3
\end{eqnarr}
\subsection{Partikuläre Lösung der inhomogenen DGL}
\label{aufsuchenpartloes}
