\section{Lineare DGL 2. Ordnung}
(mit konstanten Koeffizienten)

Allgemeine Form einer lin. DGL 2. Ordnung mit konst. Koeff:
\begin{equation*}
    y'' + a \cdot y' + b\cdot y = g(x)
\end{equation*}

\textbf{Satz}
\begin{equation*}
    y(x) = y_h(x) + y_p(x)
\end{equation*}
Analog zur linearen DGL 1. Ordnung, Lösung in zwei Schritten:
\begin{outline}
    \1 $y_h$ finden: Formel aus \ref{algloeshomdgl}
    \1 $y_p$ finden: Tabelle verwenden (\ref{aufsuchenpartloes})
\end{outline}

\subsection{Allgemeine Eigenschaften der homogenen DGL}
\begin{equation*}
    y'' + a\cdot y' + b\cdot y = 0 \hspace{2em} (*)
\end{equation*}

\underline{Satz}
\begin{outline}    
    \1[\circled{1}] Ist $y(x)$ eine Lösung von (*), dann auch $C\cdot
    y(x),C\in\mathbb{R}$
    \1[\circled{2}] Sind $y_1(x)$ und $y_2(x)$ Lösngen von (*), dann auch $y_1(x)+y_2(x)$
    \1[\circled{3}] Ist $y(x)=u(x)+j\cdot v(x)$ eine \emph{komplexe} Lösung von (*), dann auch $u(x)$ und $v(x)$\end{outline}

\bsp{Beispiel}
\begin{equation*}
    y'' + \omega ^2 \cdot y =0, \hspace{2em} (\omega >0)
\end{equation*}

$y = \sin(\omega x)$ ist Lösung:
\begin{eqnarr}
    \left[ \sin(\omega x) \right]'' + \omega ^2 \left[ \sin(\omega x) \right] &=& 0\\
    -\omega^2\sin(\omega x)  + \omega ^2 \sin(\omega x) &=& 0\\
\end{eqnarr}

$y = \cos(\omega x)$ ist Lösung:
\begin{eqnarr}
    \left[ \sin(\omega x) \right]'' + \omega ^2 \left[ \sin(\omega x) \right] &=& 0\\
    -\omega^2\sin(\omega x)  + \omega ^2 \sin(\omega x) &=& 0\\
\end{eqnarr}

Aus \circled{1} folgt, dass $y_1=C_1\cdot\sin(\omega x)$ und $y_2=C_2\cdot\cos(\omega x)$ ebenfalls Lösungen sind.\\
Aus \circled{2} folgt, dass $y=y_1+y_2$ ebenfalls eine Lösung ist

\begin{equation*}
    \left[ C_1\cdot\sin(\omega x)+C_2\cos(\omega x) \right]'' + \omega^2\cdot\left[ C_1\cdot\sin(\omega x)+C_2\cos(\omega x) \right] =0
\end{equation*}

$\Rightarrow y= C_1\cdot\sin(\omega x)+C_2\cos(\omega x) $ hat 2 Parameter
($C_1$ und $C_2$) und ist deshalb nicht nur eine, sondern die allgemeine Lösung.

Wenn $\sin(\omega x)$ eine Lösung ist, dann sind auch $C_1\sin(\omega x)$
und $C_2\sin(\omega x)$ Lösungen. Ist $y = C_1\sin(\omega x)+
C_2\sin(\omega x)$ die allgemeine Lösung?

\bsp{Def} Zwei Funktionen heissen \underline{linear unabhängig}, falls die eine
\underline{nicht} vielfaches der anderen ist.

\bsp{Beispiele}
\begin{outline}
    \1 $\left.\begin{array}{l}y_1=\sin x\\y_2=\cos x\end{array}\right\}$ linear
        unabhängig
    \1 $\left.\begin{array}{l}y_1=4 x\\y_2=2 x\end{array}\right\}$ linear
        abhängig
    \1 $\left.\begin{array}{l}y_1=\sin^2 x\\y_2=2-2\cdot\cos^2 x\end{array}\right\}$ linear abhängig
\end{outline}

\underline{Satz}

Für $y''+a\cdot y' +b\cdot y =0$:\\
Sind $y_1(x)$ und $y_2(x)$ linear unabhängige, partikuläre Lösungen, dann
ist 
\begin{equation*}
    C_1\cdot y_1(x) +C_2\cdot y_2(x), \hspace{2em} (C_1,C_2\in\mathbb{R})
\end{equation*}
die allgemeine Lösung.

\bsp{Bsp} $y'' - 4y' -5y=0$ hat die Lösungen
\begin{eqnarr}
    y_1(x) &=& e^{5x} \\
    y_2(x) &=& e^{-x} \\
\end{eqnarr}
Die Funktionen $y_1(x)$ und $y_2(x)$ sind linear unabhängig $\Rightarrow$
\begin{equation*}
y(x) = C_1\cdot e^{5x}+C_2\cdot e^{-x}
\end{equation*}
ist die allgemeine Lösung.

\subsection{Allgemeine Lösung der homogenen DGL} \label{algloeshomdgl}
Gegeben:
\begin{equation*}
    y''+a\cdot y' + b\cdot y = 0
\end{equation*}
Gesucht: 2 partikuläre, linear unabhängige Lösungen $y_1$ und $y_2$.

Ansatz: $y=e^{\lambda x}$ ($\lambda$: Parameter $\in\mathbb{C}$)

\begin{eqnarr}
    \left[e^{\lambda x}\right]'' + a \cdot \left[e^{\lambda x}\right]' + 
    b\cdot \left[e^{\lambda x}\right] &=& 0 \\
    \lambda^2e^{\lambda x} + a\cdot \lambda \cdot e^{\lambda x} + 
    b\cdot e^{\lambda x} &=& 0 \\
    e^{\lambda x} \cdot \left( \lambda^2+ a\cdot \lambda + b\right) &=& 0 \\
\end{eqnarr}
\begin{equation*}
    \lambda^2+ a\cdot \lambda + b = 0 \hspace{2cm}\mbox{Charakteristische Gleichung der DGL}
\end{equation*}
\begin{equation*}
    \lambda_{1,2} = \frac{-a\pm\sqrt{a^2-4b}}{2}
\end{equation*}




\subsection{Partikuläre Lösung der inhomogenen DGL} \label{aufsuchenpartloes}
