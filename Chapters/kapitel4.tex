\section{Lineare DGL 2. Ordnung}
(mit konstanten Koeffizienten)

Allgemeine Form einer lin. DGL 2. Ordnung mit konst. Koeff:
\begin{equation*}
    y'' + a \cdot y' + b\cdot y = g(x)
\end{equation*}

\textbf{Satz}
\begin{equation*}
    y(x) = y_h(x) + y_p(x)
\end{equation*}
Analog zur linearen DGL 1. Ordnung, Lösung in zwei Schritten:
\begin{outline}
    \1 $y_h$ finden: Formel aus \ref{algloeshomdgl}
    \1 $y_p$ finden: Tabelle verwenden (\ref{aufsuchenpartloes})
\end{outline}

\subsection{Allgemeine Eigenschaften der homogenen DGL}
\begin{equation*}
    y'' + a\cdot y' + b\cdot y = 0 \hspace{2em} (*)
\end{equation*}

\underline{Satz}
\begin{outline}    
    \1[\circled{1}] Ist $y(x)$ eine Lösung von (*), dann auch $C\cdot
    y(x),C\in\mathbb{R}$
    \1[\circled{2}] Sind $y_1(x)$ und $y_2(x)$ Lösngen von (*), dann auch $y_1(x)+y_2(x)$
    \1[\circled{3}] Ist $y(x)=u(x)+j\cdot v(x)$ eine \emph{komplexe} Lösung von (*), dann auch $u(x)$ und $v(x)$\end{outline}

\bsp{Beispiel}
\begin{equation*}
    y'' + \omega ^2 \cdot y =0, \hspace{2em} (\omega >0)
\end{equation*}

$y = \sin(\omega x)$ ist Lösung:
\begin{eqnarr}
    \left[ \sin(\omega x) \right]'' + \omega ^2 \left[ \sin(\omega x) \right] &=& 0\\
    -\omega^2\sin(\omega x)  + \omega ^2 \sin(\omega x) &=& 0\\
\end{eqnarr}

$y = \cos(\omega x)$ ist Lösung:
\begin{eqnarr}
    \left[ \sin(\omega x) \right]'' + \omega ^2 \left[ \sin(\omega x) \right] &=& 0\\
    -\omega^2\sin(\omega x)  + \omega ^2 \sin(\omega x) &=& 0\\
\end{eqnarr}

Aus \circled{1} folgt, dass $y_1=C_1\cdot\sin(\omega x)$ und $y_2=C_2\cdot\cos(\omega x)$ ebenfalls Lösungen sind.\\
Aus \circled{2} folgt, dass $y=y_1+y_2$ ebenfalls eine Lösung ist

\begin{equation*}
    \left[ C_1\cdot\sin(\omega x)+C_2\cos(\omega x) \right]'' + \omega^2\cdot\left[ C_1\cdot\sin(\omega x)+C_2\cos(\omega x) \right] =0
\end{equation*}

$\Rightarrow y= C_1\cdot\sin(\omega x)+C_2\cos(\omega x) $ hat 2 Parameter
($C_1$ und $C_2$) und ist deshalb nicht nur eine, sondern die allgemeine Lösung.

Wenn $\sin(\omega x)$ eine Lösung ist, dann sind auch $C_1\sin(\omega x)$
und $C_2\sin(\omega x)$ Lösungen. Ist $y = C_1\sin(\omega x)+
C_2\sin(\omega x)$ die allgemeine Lösung?

\bsp{Def} Zwei Funktionen heissen \underline{linear unabhängig}, falls die eine
\underline{nicht} vielfaches der anderen ist.

\bsp{Beispiele}
\begin{outline}
    \1 $\left.\begin{array}{l}y_1=\sin x\\y_2=\cos x\end{array}\right\}$ linear
        unabhängig
    \1 $\left.\begin{array}{l}y_1=4 x\\y_2=2 x\end{array}\right\}$ linear
        abhängig
    \1 $\left.\begin{array}{l}y_1=\sin^2 x\\y_2=2-2\cdot\cos^2 x\end{array}\right\}$ linear abhängig
\end{outline}

\underline{Satz}

Für $y''+a\cdot y' +b\cdot y =0$:\\
Sind $y_1(x)$ und $y_2(x)$ linear unabhängige, partikuläre Lösungen, dann
ist 
\begin{equation*}
    C_1\cdot y_1(x) +C_2\cdot y_2(x), \hspace{2em} (C_1,C_2\in\mathbb{R})
\end{equation*}
die allgemeine Lösung.

\bsp{Bsp} $y'' - 4y' -5y=0$ hat die Lösungen
\begin{eqnarr}
    y_1(x) &=& e^{5x} \\
    y_2(x) &=& e^{-x} \\
\end{eqnarr}
Die Funktionen $y_1(x)$ und $y_2(x)$ sind linear unabhängig $\Rightarrow$
\begin{equation*}
y(x) = C_1\cdot e^{5x}+C_2\cdot e^{-x}
\end{equation*}
ist die allgemeine Lösung.

\subsection{Allgemeine Lösung der homogenen DGL} \label{algloeshomdgl}
Gegeben:
\begin{equation*}
    y''+a\cdot y' + b\cdot y = 0
\end{equation*}
Gesucht: 2 partikuläre, linear unabhängige Lösungen $y_1$ und $y_2$.

Definition: 
\begin{equation*}
    \boxed{\delta = \frac{a}{2}, \hspace{1em} \omega_0 = \sqrt{b}}
\end{equation*}
Die DGL mit $\delta$ und $\omega_0$ statt $a$ und $b$:
\begin{equation*}
    y''+2\delta\cdot y' + \omega_0^2\cdot y = 0
\end{equation*}

Ansatz: $y=e^{\lambda x}$ ($\lambda$: Parameter $\in\mathbb{C}$)

\begin{eqnarr}
    \left[e^{\lambda x}\right]'' + 2\delta \cdot \left[e^{\lambda x}\right]' + 
    \omega_0^2\cdot \left[e^{\lambda x}\right] &=& 0 \\
    \lambda^2e^{\lambda x} + 2\delta\cdot \lambda \cdot e^{\lambda x} + 
    \omega_0^2\cdot e^{\lambda x} &=& 0 \\
    e^{\lambda x} \cdot \left( \lambda^2+ 2\delta\cdot \lambda + \omega_0^2
    \right) &=& 0 \\
    \lambda^2+ 2\delta\cdot \lambda + \omega_0^2 &=&  0\\
\end{eqnarr}
Charakteristische Gleichung der DGL, mit der Lösung
\begin{equation*}
    \lambda_{1,2} = \frac{-a\pm\sqrt{a^2-4b}}{2} =
    -\delta\pm\sqrt{\delta^2-\omega_0^2}
\end{equation*}
Fall 1: $\boxed{\delta^2>\omega_0^2}$

$\Rightarrow$ 2 reelle Lösungen:
\begin{eqnarr}
    \lambda_1 &=& -\delta+\sqrt{\delta^2-\omega_0^2} \\
    \lambda_2 &=& -\delta-\sqrt{\delta^2-\omega_0^2} \\
\end{eqnarr}
Zwei Lösungen gemäss Ansatz: 
\begin{eqnarr}
    y_1 &=& e^{\lambda_1 \cdot x}\\
    y_2 &=& e^{\lambda_2 \cdot x}\\
\end{eqnarr}
Allgemeine Lösung (\circled{1} und \circled{2})
\begin{equation*}
    \boxed{y(x)=C_1\cdot e^{\lambda_1 \cdot x}+C_2\cdot e^{\lambda_2 \cdot x}}
\end{equation*}
\bsp{Beispiel}
\begin{equation*}
    y''+\tikzmark{a}2y'-\tikzmark{b}8y=0
\end{equation*}
\begin{center}
    \begin{tikzpicture}[overlay,remember picture]
        \node at (-1,0) (ta) {$\delta=1$};
        \node at (1,0) (tb) {$\omega_0^2=-8$};
        \draw[->] (ta) to (a);
        \draw[->] (tb) to (b);
    \end{tikzpicture}
\end{center}
Charakteristische Gleichung:
\begin{equation*}
    \lambda^2 +2\lambda -8 = 0
\end{equation*}
\begin{eqnarr}
    \lambda_{1,2} &=& -\delta\pm\sqrt{\delta^2-\omega_0^2}\\
    &=& -1 \pm\sqrt{1-(-8)}\\
    &&\\
    \lambda_1&=& -4 \\
    \lambda_2 &=& 2\\
\end{eqnarr}
\begin{equation*}
    y(x)=C_1\cdot e^{-4x}+C_2\cdot e^{2x}
\end{equation*}

Fall 2: $\boxed{\delta^2-\omega_0^2=0}$
\begin{equation*}
    \lambda = -\delta\pm\sqrt{\delta^2-\omega_0^2} = -\delta
\end{equation*}
Problem: Nur eine Lösung $y_1(x)=e^{-\delta\cdot x}$

\bsp{Behauptung}
$y_2(x)=x\cdot e^{\lambda x}$ ist auch eine Lösung

\bsp{Beweis:}
\begin{eqnarr}
    \left[ x\cdot e^{\lambda x} \right]'' + 2\delta\cdot\left[ x\cdot e^{\lambda x}
    \right]'+\omega_0^2\cdot \left[ x\cdot e^{\lambda x} \right] &=& 0\\
    \left(1\cdot e^{\lambda x}+\lambda x\cdot e^{\lambda x} \right)' +
    2\delta\cdot\left(1\cdot e^{\lambda x}+\lambda x\cdot e^{\lambda x}
    \right)+\omega_0^2\cdot \left( x\cdot e^{\lambda x} \right) &=& 0\\
    \left(\lambda\cdot e^{\lambda x}+\lambda \cdot e^{\lambda x} +\lambda^2 x
    \cdot e^{\lambda x} \right) + 2\delta\cdot\left(1\cdot e^{\lambda x}+\lambda
    x\cdot e^{\lambda x} \right)+\omega_0^2\cdot \left( x\cdot e^{\lambda x} \right) &=& 0\\
    \underbrace{\left( 2\lambda + 2\delta \right)}_{=0,~\lambda=-\delta}\cdot e^{\lambda x} +
    \underbrace{\left( \lambda^2+2\delta\lambda
    +\omega_0^2\right)}_{=0\text{, (Char. Glg)}}\cdot x\cdot e^{\lambda x} &=& 0
\end{eqnarr}
Zwei Lösungen $y_1$ und $y_2\Rightarrow$ Allgemeine Lösung
\begin{equation*}
    \boxed{y(x)=C_1\cdot e^{\lambda x}+C_2\cdot x\cdot e^{\lambda x}}
\end{equation*}

\bsp{Beispiel:}
\begin{equation*}
    y''-\tikzmark{a}8y'+\tikzmark{b}16y=0
\end{equation*}
\begin{center}
    \begin{tikzpicture}[overlay,remember picture]
        \node at (-1,0) (ta) {$\delta=-4$};
        \node at (1,0) (tb) {$\omega_0^2=16$};
        \draw[->] (ta) to (a);
        \draw[->] (tb) to (b);
    \end{tikzpicture}
\end{center}
\begin{eqnarr}
    \lambda_1 &=& -\delta + \sqrt{\delta^2-\omega_0^2} \\
    \lambda_2 &=& -\delta - \sqrt{\delta^2-\omega_0^2} \\
    \lambda_1 = \lambda_2&=& 4 =\lambda
\end{eqnarr}
$\Rightarrow$ Allgemeine Lösung:
\begin{equation*}
    y(x)=C_1\cdot e^{4x} + C_2\cdot x\cdot e^{4x}
\end{equation*}

Fall 3: $\boxed{\delta^2-\omega_0^2<0}$
\begin{eqnarr}
    \lambda_{1,2} &=&  -\delta \pm
    \sqrt{
        \vphantom{\omega_d^2}
        \smash{\underbrace{\delta^2-\omega_0^2}_{-\omega_d^2}}
    }\\
\end{eqnarr}
\begin{equation*}
    \boxed{\omega_d^2 = \omega_0^2 - \delta^2}
\end{equation*}
Zwei komplexe Lösungen
\begin{equation*}
    \lambda_{1,2} = -\delta\pm j\cdot\omega_d
\end{equation*}
\begin{equation*}
    \tilde{y}_1(x)=e^{\lambda_1\cdot x},\hspace{2em} 
    \tilde{y}_2(x) = e^{\lambda_2 \cdot x}
\end{equation*}
Das sind zwei ireelle Lösungen. Die Kombination $y=\tilde{C}_1\cdot
\tilde{y}_1+\tilde{C}_2\cdot\tilde{y}_2$ ist die allgemeine Lösung der DGL. Ist es
möglich, die allgemeine Lösung auch als reelle Lösung zu schreiben?
\begin{eqnarr}
    y_1(x) &=&  e^{\left( -\delta+j\omega_d \right)\cdot x} \\
    &=& e^{-\delta x}\cdot e^{j\omega_d x}\\
    &=& e^{-\delta x}\left( \cos(\omega_d x)+j\cdot\sin(\omega_d x) \right)\\
    \text{Re}\left[ \tilde{y}_1(x) \right] &=& e^{-\delta x}\cdot \cos(\omega_d x) \\
    \text{Im}\left[ \tilde{y}_1(x) \right] &=& e^{-\delta x}\cdot \sin(\omega_d x) \\
\end{eqnarr}
$\Rightarrow$ Aus \circled{3} folgt, dass $ y_1=\text{Re}\left[ \tilde{y}_1(x)
\right]$ und $ y_2=\text{Im}\left[ \tilde{y}_1(x) \right]$ auch Lösungen der 
DGL sind. $\Rightarrow$

Allgemeine Lösung:
\begin{equation*}
    \boxed{
        y(x) = C_1 \cdot e^{-\delta x}\cdot \cos(\omega_d x)
             + C_2 \cdot e^{-\delta x}\cdot \sin(\omega_d x)
    }
\end{equation*}

\bsp{Beispiel}
\begin{equation*}
    y''+\tikzmark{a}4y'+\tikzmark{b}13y=0
\end{equation*}
\begin{center}
    \begin{tikzpicture}[overlay,remember picture]
        \node at (-1,0) (ta) {$\delta=2$};
        \node at (1,0) (tb) {$\omega_0^2=13$};
        \draw[->] (ta) to (a);
        \draw[->] (tb) to (b);
    \end{tikzpicture}
\end{center}
\begin{eqnarr}
    \delta^2-\omega_0^2 &=& 4-13\\
    &=& -9\\
    &<& 0\Rightarrow \text{ Fall 3}\\
\end{eqnarr}
\begin{eqnarr}
    \delta &=& 2\\
    \omega &=& \sqrt{\omega_0^2-\delta^2}\\
    &=& \sqrt{13-4} = 3
\end{eqnarr}
Die allgemeine Lösung ist also
\begin{equation*}
    y(x) = C_1\cdot e^{-2x}\cdot \cos(3x)
         + C_2\cdot e^{-2x}\cdot \sin(3x)
\end{equation*}

\subsection{Partikuläre Lösung der inhomogenen DGL}
\label{aufsuchenpartloes}
\begin{equation*}
    y(x)''+2\delta\cdot y(x)'+\omega_0^2\cdot y(x) = g(x)
\end{equation*}
Theorie: Ansatz für $y(x)$ aus der Tabelle im Papula S. 276

\bsp{Beispiel}
\begin{equation*}
    y''+y'-2y=10x+1
\end{equation*}
\underline{homogen}
\begin{equation*}
    y''+\tikzmark{a}y'-\tikzmark{b}2y=0
\end{equation*}
\begin{center}
    \begin{tikzpicture}[overlay,remember picture]
        \node at (-1,0) (ta) {$\delta=\frac{1}{2}$};
        \node at (1,0) (tb) {$\omega_0^2=-2$};
        \draw[->] (ta) to (a);
        \draw[->] (tb) to (b);
    \end{tikzpicture}
\end{center}
$\delta^2-\omega_0^2=\frac{9}{4}>0\Rightarrow$ Fall 1
\begin{equation*}
    \lambda_{1,2}=-\delta\pm\sqrt{\delta^2-\omega_0^2} =
    \frac{-1}{2}\pm\frac{3}{2}
\end{equation*}
$\lambda_1 = 1,~\lambda_2=-2$
\begin{equation*}
    y_h = C_1 \cdot e^x + C_2\cdot e^{-2x}
\end{equation*}

\underline{inhomogen}
Ansatz $(b\neq 0)$: $y_p = A\cdot x+B$
\begin{eqnarr}
    \left[ Ax+B \right]''+\left[ Ax+B \right]'-2\left[ Ax+B \right]&=& 10x+1\\
    A-2Ax-2B&=& 10x+1\\
    \\
    \left|\begin{array}{c}
        -2A=10\\A-2B=1
    \end{array}\right|
    &\Rightarrow& A=-5, \hspace{1em} B=-3
\end{eqnarr}
$y_p = -5x-3\Rightarrow$
\begin{equation*}
    y(x) = y_h+y_p = C_1 \cdot e^x + C_2\cdot e^{-2x} -5x -3
\end{equation*}

\bsp{Beispiel}
\begin{equation*}
    y''+y=\sin(x)
\end{equation*}

\underline{homogen}
\begin{equation*}
    y''\tikzmark{a}+\tikzmark{b}y=0
\end{equation*}
\begin{center}
    \begin{tikzpicture}[overlay,remember picture]
        \node at (-1,0) (ta) {$\delta=0$};
        \node at (0.5,0) (tb) {$\omega_0^2=1$};
        \draw[->] (ta) to (a);
        \draw[->] (tb) to (b);
    \end{tikzpicture}
\end{center}
$\delta^2-\omega_0^2=-1>0\Rightarrow$ Fall 3
\begin{eqnarr}
    \omega_d &=& \sqrt{\omega_0^2-\delta^2} = \sqrt{1-0} = 1\\
    \lambda_{1,2} &=& -\delta \pm j\omega_d = \pm j\\
\end{eqnarr}
\begin{eqnarr}
    y_h &=&  C_1 \cdot e^{0\cdot x}\cdot\cos(1\cdot x)
           + C_2 \cdot e^{0\cdot x}\cdot\sin(1\cdot x)\\
        &=& C_1\cos(x)+C_2\sin(x)
\end{eqnarr}

\underline{inhomogen}
\begin{equation*}
    g(x) = \sin(\tikzmark{a}x)
\end{equation*}
\begin{center}
    \begin{tikzpicture}[overlay,remember picture]
        \node at (0,0) (ta) {$\beta=1$};
        \draw[->] (ta) to (a);
    \end{tikzpicture}
\end{center}

$j\beta = j$ \emph{ist} Lösung der charakteristischen Gleichung! Also Ansatz
\begin{equation*}
    y_p(x) = x\cdot \left( A\cdot\sin(x)+B\cdot\cos(x) \right)
\end{equation*}
\begin{eqnarr}
    [\ldots]''+[\ldots] &=& \sin(x)\\
    &\vdots&\\
    A=0 && B = -\frac{1}{2}
\end{eqnarr}
\begin{eqnarr}
    y_p(x) &=&  x\cdot \left( 0-\frac{1}{2}\cdot\cos(x) \right)\\
    &=& -\frac{1}{2}\cdot x\cdot \cos(x)
\end{eqnarr}
Also
\begin{equation*}
    y(x) = y_h+y_p = C_1\cos(x)+C_2\sin(x) -\frac{1}{2}\cdot x\cdot \cos(x)
\end{equation*}
