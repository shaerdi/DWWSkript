\section{Definition, Klassifikation, Beispiele}
\definition{Differentialgleichung (DGL)}{
Gleichung, deren Unbekannte eine Funktion ist, und die Ableitungen der gesuchten Funktion enthält.}

\bsp{Beispiele}
\begin{IEEEeqnarray*}{rCl}
    y''\left( x \right)+2y\left( x \right) &=& 0 \\
    y''+2y&=& x^{2} \\
    \ddot{y}+2y &=& 0 \\
    \frac{\partial^2f}{\left( \partial x_1 \right)^{2}} + 
    \frac{\partial^2f}{\left( \partial x_2 \right)^{2}} &=& 0
\end{IEEEeqnarray*}

\definition{Spezielle / partikuläre Lösung}{
\underline{Eine} Funktion, die die DGL erfüllt.}
\definition{Allgemeine Lösung}{
Funktion mit Parametern, die alle speziellen Lösungen parametrisiert
(beinhaltet).}

Zum Festlegen der Parameter einer allgemeinen Lösungen werden Zusatz\-bedingungen
benötigt. Je nachdem ob diese im Ort oder in der Zeit ge\-ge\-ben sind, 
wird eine DGL mit Zusatzbedingungen ein \emph{Anfangswertproblem} oder ein
\emph{Randwertproblem} genannt.

\bsp{Beispiel}
\begin{equation*}
    \ddot{y}=-y
\end{equation*}
\begin{IEEEeqnarray*}{rrCl}
    \mbox{allgemeine Lsg:\hspace{2em}} & y\left( t \right) &=& A\sin\left(
    t+\phi \right) \\
    \mbox{Zusatzbedingungen: \hspace{2em}} & y\left( 0 \right) &=& 0
    \mbox{ und } y\left( \pi/2 \right) = 2 \\
    \mbox{spezielle Lsg:\hspace{2em}} & y\left( t \right) &=& 2\sin\left( t \right)
\end{IEEEeqnarray*}

Notation
\begin{IEEEeqnarray*}{rCl}
    y'&:& \mbox{ Ableitung der Funktion y nach der Variablen x
    (meistens Ort)}\\ 
    \dot{y}&:& \mbox{ Ableitung der Funktion y nach der Variablen t
    (meistens Zeit)}\\ 
    y^{(n)} &:& \mbox{ n-te Ableitung der Funktion y}\\
    y^n &:& \mbox{ n-te Potenz der Funktion y}\\
    \frac{\partial f}{\partial x} &:& \mbox{ Ableitung der Funktion f nach x}
\end{IEEEeqnarray*}
